\phantom.
\vspace{3cm}
\section{Introduction}
The goal of theoretical condensed matter and materials physics is a precise description of the fundamental properties of matter. In view of technological applications, particularly in photovoltaics and optoelectronics, the optical properties like  the absorption behavior of a material are of special interest \cite{CARDONA2011125}. They are determined by the excitations of coupled electron-hole pairs, excitons, which are formed due to the attractive Coulomb potential between the particles. Of particular importance is the binding energy of the excitonic ground state, which  determines the optical absorption onset. The interaction strength between electron and hole and, thus, the binding energy, are affected by the formation of polarization clouds around the charge carriers. This process, known as screening,  plays a central role in the exciton formation, and is therefore  a major object of research\cite{bechstedt2016many,reining_int_el}.\par 
Many materials studied owing to their technological applicability are polar. Classical examples are  GaN and MgO\cite{draxl_gan,fuchs_08}, but in the last decades also more complex materials like  $\text{Ga}_2\text{O}_3$\cite{ga_ex1, ga_ex2} and various perovskites\cite{exciton_perovskites}  have attracted attention due to their outstanding properties. At variance with non-polar semiconductors like silicon, the coupling between electrons and phonons, collective vibrational modes of the crystal lattice, plays a crucial role in the formation of excitons. For large wavelengths, the polar nature of the crystal gives rise to macroscopic electric fields, that can directly couple to charge carriers. The attraction between electron and hole is therefore not only screened by electrons but also by the lattice. However, the lattice polarization is not always able to follow the exciton formation. In materials with large excitonic binding energies, the exciton formation is so fast that the lattice is unable to follow and does not contribute to the screening. In this case, the screening is solely given by the electronic  contribution. By contrast, the lattice can fully contribute to the screening if the binding energy is small and the exciton formation is slow. In all cases between those extremes, the lattice contributes partially to the screening\cite{bechstedt2016many}. \par 
For understanding the optical properties and designing novel materials for optoelectronic devices, the precise simulation of excitonic effects is indispensable. In this context, \textit{ab-initio} or first-principles methods, describing physical processes on a fundamental quantum-mechanical level, are highly relevant. Those methods require no empirical information about the system and are therefore widely applicable and more predictive than parameterized methods relying on experimental input. A rigorous theoretical framework allowing for first-principles calculations of both electron-hole and electron-phonon interactions is given by a many-body formulation based on quantum field theory. In particular, the solution of the Bethe-Salpeter equation (BSE) has become the state-of-the-art technique in order to deal with electron-hole interactions in periodic systems and molecules\cite{strinati1988application,rohlf_louie_2000}.\par 
However, many BSE implementations (see, \textit
{e.g.}, \cite{Vorwerk_2019,simple_x}) consider only electronic screening effects, that can be calculated \textit{ab initio} in the random-phase approximation\cite{random_phase}. Regarding the impact of electron-phonon coupling on the electron-hole interaction, so far no description from first principles exists, although the problem is well-known in the literature\cite{bechstedt2016many,cardona2005fundamentals}. Therefore, the effects of lattice screening are either completely neglected or treated at a certain level of approximation. The simplest approach consists of replacing the pure electronic dielectric constant  by an effective dielectric constant obtained from fitting the calculated binding energy to an experimental value\cite{fuchs_08}. A~more refined strategy used in the literature, is the use of models for the dielectric function. Nevertheless,  simplifying assumptions are needed, which include the restriction to the long-wave-length limit and  the coupling to  only specific phonon modes\cite{highly_ionic,Bechstedt}. Despite the fact that these approaches improve the theoretical predictions of exciton binding energies in a quantitative way, it is clear that a more fundamental description is desirable. Especially for complex materials like  $\text{Ga}_2\text{O}_3$, having multiple phonon modes, more general models are crucial to correctly capture the physical processes taking place.\par
The aim of this work is to provide a first-principles approach describing the influence of electron-phonon coupling on excitonic binding energies in polar insulators and semiconductors. This is achieved by deriving a phonon contribution to the screened Coulomb interaction, which is  then included in the solution of the BSE. Here, a central step is the explicit treatment of dynamical effects. The implementation into the all-electron code \exciting{}\cite{exciting} enables the numerical evaluation of the developed approach.\par
The thesis is structured as follows: Chapter~\ref{dft_sec} is devoted to the calculation of the electronic ground state using density-functional theory, and the computation of phonons in polar materials. In Chapter~\ref{mbpt_sec}, the central ideas of many-body perturbation theory  and the calculation of excited states via the BSE are presented. Here, also the fundamentals of electron-phonon interactions are reviewed. Chapter~\ref{sec_phass}  begins with a general derivation of the phonon contribution to the screened Coulomb interaction  and its effect on the solutions of the BSE. The approach is applied to  polar materials and the implementation into \exciting{} is presented. In  Chapter~\ref{result_sec}, the results for various polar semiconductors and insulators are discussed. The values computed  within the first-principles approach are compared with results obtained by the Wannier-Mott model and with experimental data.
